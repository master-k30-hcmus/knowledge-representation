\documentclass[a4paper]{article}
\usepackage[utf8]{vietnam}
\usepackage{graphicx}
\usepackage{amssymb}
\usepackage{amsmath}
\usepackage{scrextend}
\usepackage{tcolorbox}
\usepackage{setspace}
\usepackage{times}
\usepackage{listings}
\usepackage{tvietlistings}
\usepackage{xcolor}
\usepackage{indentfirst}
\setlength{\parindent}{0.5cm}

%New colors defined below
\definecolor{codegreen}{rgb}{0,0.6,0}
\definecolor{codegray}{rgb}{0.5,0.5,0.5}
\definecolor{codepurple}{rgb}{0.58,0,0.82}
\definecolor{backcolour}{rgb}{0.95,0.95,0.92}

%Code listing style named "mystyle"
\lstdefinestyle{mystyle}{
	backgroundcolor=\color{backcolour},   commentstyle=\color{codegreen},
	keywordstyle=\color{magenta},
	numberstyle=\tiny\color{codegray},
	stringstyle=\color{codepurple},
	basicstyle=\ttfamily\footnotesize,
	breakatwhitespace=false,         
	breaklines=true,                 
	captionpos=b,                    
	keepspaces=true,                 
	numbers=left,                    
	numbersep=5pt,                  
	showspaces=false,                
	showstringspaces=false,
	showtabs=false,                  
	tabsize=2
}

%"mystyle" code listing set
\lstset{style=mystyle}

\onehalfspacing
\textwidth=16 cm
\oddsidemargin=0 cm
\evensidemargin=1 cm
\topmargin= -1 cm

\renewcommand\thesection{Câu \alph{section}.}
\renewcommand\thesubsection{\roman{subsection}.}

\title{BIỄU DIỄN TRI THỨC\\ Bài tập 3}
\author{Nhóm 07}
\date{June 12th, 2021}

\begin{document}
	\maketitle
	\begin{center}
		\LARGE{\textbf{Bài 3\\Thu thập tri thức}}
		
		\Large{\textbf{Lý thuyết đồ thị}}
		
		\Large{\textbf{Bài toán tìm đường đi ngắn nhất}}
	\end{center}
	
	\section{Xác định nguồn tri thức} 
	
	Bảng tri thức thu thập được trình bày trong tập tin \textit{BT3-Nhom07-CauA.pdf}.
	
	\section{Mô hình biểu diễn tri thức} 
	Mô hình $(C, R)$:
	
	\begin{itemize}
		\item $C$ =  tập các yếu tố tri thức ở câu a,
		với $c \in C: (\text{yếu tố tri thức, nội dung})$
		
		\item $R$ = Tập hợp mối quan hệ giữa các $c \in C$ với nhau, 
		với $R$ = {"is-a", "has-a"}
	\end{itemize}
	
	\section{Tổ chức lưu trữ miền tri thức} 
	
	Tổ chức lưu trữ miền tri thức trong hệ quản trị cơ sở dữ liệu \texttt{(CSDL) MS SQL Server} với cấu trúc bảng như sau:
	
	\begin{itemize}
		\item Bảng \textbf{\textbf{KNOWLEDGE\_REPRESENTATION}}: mỗi dòng định danh một yếu tố tri thức trong miền tri thức lý thuyết đồ thị.
		
		\begin{itemize}
			\item \texttt{id} là dãy số duy nhất tương ứng với một yếu tố tri thức 
			\item \texttt{name} tên yếu tố tri thức
			\item \texttt{content} nội dung yếu tố tri thức được trình bày
			\item \texttt{keypharse} cụm từ khóa được sử dụng để tìm kiếm yếu tố tri thức, thường là tên và thành phần của yếu tố tri thức.
			\item \texttt{related\_keypharse} cụm từ khóa của những yếu tố tri thức có liên quan tới yếu tố tri thức đang được xét.
		\end{itemize} 
		
		\begin{tabular}{|c|l|}
			\hline
			\multicolumn{2}{|c|}{\textbf{KNOWLEDGE\_REPRESENTATION}} \\
			\hline
			\textbf{PK} & \texttt{\textbf{id int NOT NULL}} \\
			\hline
			& \texttt{name char(50) NOT NULL} \\
			\hline
			& \texttt{content char(1000) NOT NULL} \\
			\hline
			& \texttt{keypharse char(250) NOT NULL} \\
			\hline
			& \texttt{related\_keypharse char(250) NOT NULL} \\
			\hline
		\end{tabular}
		
	\end{itemize}
	
	Miền tri thức trên cho phép giải các bài toán liên quan tới tìm kiếm tri thức và tri thức liên quan tới lý thuyết đồ thị.
	
	

\end{document}