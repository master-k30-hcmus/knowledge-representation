\section{Tổ chức lưu trữ miền tri thức} 

Tổ chức lưu trữ miền tri thức trong hệ quản trị cơ sở dữ liệu \texttt{(CSDL) MS SQL Server} với cấu trúc bảng như sau:

\begin{itemize}
	\item Bảng \textbf{\textbf{KNOWLEDGE\_REPRESENTATION}}: mỗi dòng định danh một yếu tố tri thức trong miền tri thức lý thuyết đồ thị. 
	\begin{itemize}
		\item \texttt{id} là dãy số duy nhất tương ứng với một yếu tố tri thức 
		\item \texttt{name} tên yếu tố tri thức
		\item \texttt{content} nội dung yếu tố tri thức được trình bày
		\item \texttt{keypharse} cụm từ khóa được sử dụng để tìm kiếm yếu tố tri thức, thường là tên và thành phần của yếu tố tri thức.
		\item \texttt{related\_keypharse} cụm từ khóa của những yếu tố tri thức có liên quan tới yếu tố tri thức đang được xét.
	\end{itemize} 
	
	\begin{tabular}{|c|l|}
		\hline
		\multicolumn{2}{|c|}{\textbf{KNOWLEDGE\_REPRESENTATION}} \\
		\hline
		\textbf{PK} & \texttt{\textbf{id int NOT NULL}} \\
		\hline
		& \texttt{name char(50) NOT NULL} \\
		\hline
		& \texttt{content char(1000) NOT NULL} \\
		\hline
		& \texttt{keypharse char(250) NOT NULL} \\
		\hline
		& \texttt{related\_keypharse char(250) NOT NULL} \\
		\hline
	\end{tabular}
	
\end{itemize}

Miền tri thức trên cho phép giải các bài toán liên quan tới tìm kiếm tri thức và tri thức liên quan tới lý thuyết đồ thị.

