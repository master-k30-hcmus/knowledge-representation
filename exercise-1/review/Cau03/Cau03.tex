\documentclass[a4paper]{article}
\usepackage[utf8]{vietnam}
\usepackage{scrextend}
\usepackage{graphicx}
\usepackage{times}


\begin{document}

\section*{THỰC HÀNH}	
Chọn một ứng dụng trong thực tế và cho biết:

\begin{enumerate} 
	\item Miền của tri thức. 
	\item Đối tượng sử dụng. 
	\item Yêu cầu và chức năng của hệ thống đòi hỏi vận dụng tri thức. 
	\item Phân tích và nhận định về đặc trưng của tri thức trong ứng dụng
\end{enumerate}


\section*{BÀI LÀM}
		
\subsection* {1. Miền của tri thức}
Miền tri thức bao gồm tri thức về đại số và tuyến tính.

\subsection*{2. Đối tượng sử dụng}
Đối tượng sử dụng bao gồm sinh viên các trường trung học, cao đẳng, đại học và những người có nhu cầu tìm hiểu học về đại số tuyến tính.

\subsection*{3. Yêu cầu và chức năng của hệ thống đòi hỏi vận dụng tri thức}
\textbf{Yêu cầu: }Thiết bị điện tử dùng nhập yêu cầu ( điện thoại, bàn phím máy tính), màn hình, kết nối mạng internet.\\
\textbf{Chức năng: } Nhận yêu cầu từ người dùng, sử dụng hệ thống tri thức trong ứng dụng để tính toán, biến đổi và trả ra màn hình kết quả với độ chính xác cao, lời giải chi tiết và thời gian xử lý là ngắn nhất.

 \subsection*{4. Phân tích và nhận định về đặc trưng của tri thức trong ứng dụng}

\begin{enumerate}
	
\item [4.1] Phân tích đặc trưng của tri thức
\begin{itemize}
\item Tri thức về đại số bao gồm các phép toán số học, giả phương trình, bất phương trình, vẽ đồ thị hàm số trong không gian 2 chiều.
\item Tri thức về giải tích bao gồm các phép toán tính đạo hàm, vi phân, và tích phân.
\item Tri thức về ma trận bao gồm các phép toán nhân, cộng trừ, biến đổi ma trận
\end{itemize}

\item [4.2]Nhận định về đặc trưng của tri thức
\begin{itemize}
\item Vì áp dụng từng bước giải theo những hệ tri thức được đưa vào sẵn có nên có những bước dư thừa trong tính toán.
\item Các phương pháp biểu diễn tri thức tạo ra trong hệ thống chưa bao quát được hết những trường hợp trong thực tế, do đó ứng dụng chỉ có thể giải quyết những bài toán đơn giản, chưa giải quyết được những bài toán phức tạp đòi hỏi những phép biến đổi, suy luận phức tạp.
\end{itemize}

\end{enumerate}	

\end{document}