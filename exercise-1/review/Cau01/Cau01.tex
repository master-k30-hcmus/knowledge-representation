\documentclass{article}
\usepackage{graphicx}
\usepackage{amssymb}
\usepackage{amsmath}


\usepackage[utf8]{vietnam}

\title{BIỄU DIỄN TRI THỨC\\ Bài tập 1}
\author{Nhóm 07}
\date{May 9th, 2021}

\begin{document}
	\maketitle
	\begin{center}
		\Large{\textbf{Phần A - Đọc tài liệu}}
	\end{center}
	\section*{Câu 1}
	Trình bày một ứng dụng áp dụng biểu diễn tri thức (Knowledge Representation – KR) (tham khảo Section 3) và đề xuất hướng mở rộng.
	
	\section*{BÀI LÀM}
	Ứng dụng của biễu diễn tri thức trong giáo dục là \textbf{Ứng dụng dạy kèm thông minh}.
	
	Hệ thống dạy kèm thông minh đạt hiệu quả trong một số lĩnh vực, tiêu biểu là môn Đại Số được sử dụng bởi hơn nửa triệu sinh viên Hoa Kỳ hiện nay. Ứng dụng đã cách mạng hóa nền giáo dục, bằng cách phản hồi người học mọi lúc, mọi nơi. Các phương pháp tương tác tự động như vậy đều mang lại lợi ích cho các loại hình giáo dục từ truyền thống đến các khóa học trực tuyến.
	
	\subsection*{Hướng đề xuất mở rộng}
	\begin{itemize}
		\item Xây dựng chatbox để tương tác trực tiếp với học sinh
		\item Chuyển sang hỗ trợ STEM, hệ thống dạy kèm thông minh sẽ giúp học sinh, sinh viên lập luận qua từng bước, từ đó hiểu được quá trình tư duy khoa học mà không chỉ là kết quả của bài toán. Điều này, hệ thống thông minh đòi hỏi không chỉ các kiến thức chuyên môn về  lĩnh vực đó mà hệ thống cần các kiến thức về các mô hình giải toán của học sinh/sinh viên.
		\item Hệ thống sẽ hỗ trợ kết nối giữa học sinh/sinh viên và phụ huynh. Từ đó, hệ thống sẽ giúp phụ huynh nắm bắt được tình hình họp tập của học sinh/sinh viên (phù hợp với các nước Châu Á)
		\item Hệ thống có thể tính hợp các tính năng về xử lý ảnh. Từ đó hệ thống có thể tự động trích xuất thông tin từ hình ảnh của học sinh/sinh viên chụp gửi qua hệ thống.
		
	\end{itemize}
\end{document}